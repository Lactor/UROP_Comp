\documentclass[11pt]{report}
\pagestyle{plain}
\usepackage{graphicx}
\usepackage{indentfirst}
\usepackage{float}
\usepackage[top=2.5cm, left=2.0cm, right=2.0cm, bottom=2.5cm]{geometry}
\newcommand{\unit}[1]{\ensuremath{\, \mathrm{#1}}}
%\restylefloat{table}
%\restylefloat{figure}
\usepackage{amssymb}
\usepackage{amsmath}
\usepackage{amsfonts}

\newcommand{\horline}{\begin{center} \line(1,0){470} \end{center}}


\author{Francisco Machado}
\title{Summary of the Galaxy comparison code}
\begin{document}

\maketitle

\section{Introduction}

The goal of this document is to provide an understanding of the code written to do the comparisons between galaxies.
I will cover each file and explain how the input is taken and then the workings of the program.

\section{Format\_galaxies.py}

The purpose of this file is to preprocess the simulation galaxy's SED, transforming the data from Luminosity per wavelength into AB magnitude at a distance of $10 \unit{pc}$.

\subsection{Input}

The input for this program is:
\begin{verbatim}
python3 Format_galaxies.py Input_Folder Output_Folder
\end{verbatim}

The Input\_Folder is the origin place of the files and the Output\_Folder is the directory where the modified files will be stored.

The program runs through every .txt file and runs the appropriate transformations. Right now it assumes that the file consists {\bf only} on two columns, the first one containing the wavelength ($\lambda$) and the other the Luminosity per wavelength ($L_\lambda$) at a distance of $50 \unit{Mpc}$ in SI units ($\unit{W m^{-1}}$).


Moreover the code also assumes that when a row is split by the spaces (' ') using the string.split(' ') method from python, the second column corresponds to the FILE\_SECOND\_COLUMN position in the resulting array. Changes in the input file format must be taken into consideration by changing this value.

\subsection{Data processing}

To turn the data into AB magnitudes the program follows the following steps.

Firstly the value of Luminosity per wavelength is transformed to luminosity per frequency ($L_\nu$), using the equation:
\begin{equation}
L_\nu = L_\lambda \frac{\lambda^2}{c}
\end{equation}

This value will have units of $\unit{W Hz^{-1}}$

Then the luminosity is transformed into flux ($F_\nu$) at a radius of $R = 10 \unit{pc}$, by dividing it by the area of a sphere of such radius.

\begin{equation}
F_\nu = L_\nu \frac{1}{4\pi R^2}
\end{equation}

$F_\nu$ will now have units of $\unit{W Hz^{-1}m^{-2}}$

Then we make the convertion into flux in microJanksys ($S_\nu$):
\begin{equation}
S_\nu = F_\nu \times 10^{32}
\end{equation}

The units of $S_\nu$ are $\mu\unit{Jy}$.

Then we use the formula to calulate the AB magnitude from the flux in $\mu \unit{Jy}$:

\begin{equation}
m_{AB} = 23.9 - \log_{10}\left( S_\nu \right)
\end{equation}

$m_{AB}$ will not have units but determines the apparent AB magnitude of the galaxy.

Since we calculate the flux at a distance of $10 \unit{pc}$ this apparent magnitude matches the absolute magnitude ($M_{AB}$).
\begin{equation}
M_{AB} = m_{AB}
\end{equation}

For the remainder of the article let's define the absolute magnitude of a simulation galaxy as $M_{S}$.

\subsection{Saving the Data}

The code then saves each file into a new file of the same name of the original on the Output\_Folder. Each file consists on two columns separated by a space (' ') where the first column is the wavelength ($\lambda$) and the second will be the absolute AB magnitude for that wavelength ($M_{AB}$).

\horline 

\section{FastComp.py}

This program has compares an input galaxy with a set of galaxies, by their absolute AB magnitude, using the $\chi^2$ minimization method. It returns a file with the value of $\chi^2$ for every galaxy in the comparison set, as well as drawing a graph and histogram with the same information.

\subsection{Input}

The input for this program is:
\begin{verbatim}
python3 FastComp.py Data_Folder Input_File_1 Input_File_2 ... Input_File_N
\end{verbatim}

The Input\_File\_i is a file containing the information about the galaxy we wish to compare with our set. The file must be formated in the following way. The first line must contain a single number which will correspond to the redshift of the galaxy. The remainder of the file must contain 3 columns, which will be the wavelength ($\lambda$), the flux in $\mu \unit{Jy}$ ($S_\nu$), and the error in the flux ($\delta S_\nu$) also in units of $\mu \unit{Jy}$.
Each of the columns is separated by a space (' ').

The Data\_Folder is a folder containing the galaxy files from the simulation. These are the galaxies that have been preprocessed by the Format\_galaxies.py script.

\subsection{Data Pre-Processing}

The following description is the procedure followed by the program for all the files.

The program starts by reading the information of the file and storing the redshift on the variable called (appropriately) {\bf redshift}. The remaining values are stored on an dictionary of arrays where the 'wl' corresponds to the wavelengths, 'fl', the fluxes and 'err' the error in the flux.

Then the distance in $\unit{Mpc}$ ($D_{Mpc}$) is calculated using the cosmocalc package, using the following command (included to specify the constants):
\begin{verbatim}
(cosmocalc.cosmocalc(redshift, H0=70.4, WM=0.2726, WV=0.7274))['DL_Mpc']
\end{verbatim}

Then the distance is transformed into $\unit{pc}$, (now $D_{pc}$) and used to calculate the distance modulus ($\mu$) which will then be used for calculating the variation in the magnitude of the Input\_File.

\begin{equation}
\mu = 5 \cdot ( \log_{10}( D_{pc}) - 1)
\end{equation}

Now for each data point in the Input\_File\_i we tranform the flux ($S_\nu$) into an absolute AB magnitude ($M_{AB}$) (AB magnitude measured at a distance of $10 \unit{pc}$). To do this, the flux is firstly used to calculate the apparent AB magnitude ($m_{AB}$) and then we make use of the modulus distance ($\mu$) to tran2sform from apparent to absolute.

\begin{align}
m_{AB} &= 23.9 - 2.5 \cdot \log_{10}(S_\nu)\\
M_{AB} &= m_{AB} - \mu
\end{align}

The error will also be propagated.
\begin{align*}
\delta m_{AB} &= 2.5 \frac{\delta S_\nu}{S_\nu}\\
\delta M_{AB} &= \delta m_{AB}
\end{align*}

Here we are assuming that there is no error in the distance since we have no error in the redshift which we can take into account.

For the remainder of the articles let's define the absolute magnitude of an input galaxy as $M_{I}$, and its error as $\delta M_I$.

\subsection{Comparison Method}
To compare the Input\_File\_i with the set of galaxies we use the $\chi^2$ minimization method by comparing to every galaxy in our set and select the galaxies which present the least value of $\chi^2$.

We then open a simulation galaxy file and for every input file we calculate the $\chi^2$ between the input file and that galaxy file. This way we do not open a simulation galaxy file more than once.

If the values of the wavelengths match with the ones we have for the set of galaxies, then we simply calculate the $\chi^2$ term directly:
\begin{equation}
\chi^2 = \frac{1}{N}\sum_{k=1}^N \frac{(M_S - M_I)^2}{\delta M_I}
\end{equation}

If the wavelengths do not match, then we must do an interpolation between the two neighboring points to gather the correct value we should use for $M_S$. 

Let $\lambda_S^+$ and $\lambda_S^-$ be the closest values of the wavelengths to $\lambda_I$ such that $\lambda_S^+ > \lambda_I > \lambda_S^-$.  Also let $M_S^+$ and $M_S^-$ be the values of the absolute magnitude of the simulation galaxy for wavelengths $\lambda_S^+$ and $\lambda_S^-$ respectively.

Then we can define an interpoled value for the absolute magnitude $M_S^{interpol}$ and then do the comparison with that term.

\begin{align}
M_S^{interpol} &= \frac{M_S^+ - M_S^-}{\lambda_S^+ - \lambda_S^-} (M_I - M_S^-) + M_S^-\\
\chi^2 &= \frac{1}{N} \sum_{k=1}^N \frac{ (M_S^{interpol} - M_I)^2}{\delta M_I}
\end{align}

Note, if there are values of wavelengths in the input file greater than the largest wavelength in our simulated galaxys' files or smaller than the smallest one in our simulated galaxys' files, then we have no way of doing a precise comparison with those and they are disregarded.

\subsection{Saving the Data}

The program then saves the resulting $\chi^2$ of all the file comparisons to a different file in the Results folder for each input file. Each file is composed of two columns, one is corresponds to a simulated galaxy and the other one is the $\chi^2$ of the comparison of the input file with that simulated galaxy. The values are ordered by the $\chi^2$ for quick reference.

\section{Fits\_to\_File.py}

The goal of this program is to transfrom the .fit from the Gama Survey data set to text files with a formating that is accepted by the FastComp.py program.

\subsection{Input}

The input for this file is:
\begin{verbatim}
python3 Fits_to_file.py Input_Folder Output_Folder
\end{verbatim}

The Input\_Folder is the folder contaning the fit files we wish to process to text files. The Output\_Folder is the folder where the resulting files are saved.

\subsection{Library used}

To access the information I am using the library pyfits which allows for easy acess to a fit file.

\subsection{Data Processing}

First let's list the assumptions on the fit file formating:
\begin{itemize}
\item There is only one part in the file, meaning only one header and only one set of data.
\item The redshift label on the header is 'Z'
\item The range of the wavelengths is given by the headers 'WMIN' and 'WMAX', in angstroms ($A$).
\item The wavelengths are uniformly distributed in this range
\item The first row of data corresponds to the flux in units of $10^{-17}\unit{erg\cdot s^{-1}\cdot cm^{-2} \cdot A^{-1}}$
\item The second row is the error associated with the flux in the same units.
\end{itemize}

The file starts by reading the value of the redshift and creating the possible values of the wavelengths, based on the limits in the header.

Then for each possible value of wavelength the program transforms it to SI units (meters) and reads and transforms the flux and error associated with that wave length.

The transformations of the values is done as follows (let $R_\lambda$ be the value of the flux for a wavelength $\lambda$ and $\delta R_\lambda$ its error):

\begin{align*}
L_\lambda = R_\lambda \cdot 10^{-17} \cdot 10^{10}\\
\delta L_\lambda = \delta R_\lambda \cdot 10^{-17} \cdot 10^{10}\\
\end{align*}

This transforms the units from $10^{-17}\unit{erg\cdot s^{-1}\cdot cm^{-2} \cdot A}$ to $\unit{erg\cdot s^{-1}\cdot cm^{-2} \cdot m^{-1}}$

\begin{align*}
L_\nu = L_\lambda \frac{\lambda^2}{c}\\
\delta L_\nu = \delta L_\lambda \frac{\lambda^2}{c}\\
\end{align*}

This transforms flux per wavelength ($\unit{erg\cdot s^{-1}\cdot cm^{-2} \cdot m^{-1}}$) to flux per frequency ($\unit{erg\cdot s^{-1}\cdot cm^{-2} \cdot Hz^{-1}}$)

\begin{align*}
S_\nu = L_\nu \cdot 10^{29}\
\delta S_\nu = \delta L_\nu \cdot 10^{29}\\
\end{align*}

This transforms the value from $\unit{erg\cdot s^{-1}\cdot cm^{-2} \cdot Hz^{-1}}$ to $\mu \unit{Jy}$.

\subsection{Saving data}

The program then for each file, saves a file in the output folder with the same name but .txt extension containing:
\begin{itemize}
\item First line containing the redshift of the galaxy
\item The remainder consists of lines with 3 values separated by a space containing the wavelength of light, the flux of the light in $\unit{\mu Jy}$ and the error of the flux in the same units
\end{itemize}

\end{document}
